\chapter{JavaScript 基础语法}
\section{语言基础}
\subsection{语法}

ECMAScript 语言很大程度上借鉴了 C 语言,作为 C 语言系中的一门,它和 Java 与许多相似之处,比如区分大小写。

\subsubsection{标识符}

JavaScript 对标识符组成规定如下:
\begin{itemize}
    \item 第一个字符必须是一个字母,下划线(\_)或美元符号(\$)。
    \item 剩下的字符可以是字母,数字,下划线(\_)或美元符号(\$)。
\end{itemize}

标识符中的字符是可扩展的 ASCII 中的字母,也可以是 Unicode 的字母字符,但不建议使用 Unicode 字符。

\fbox{
    \parbox{0.87\textwidth}{
        \begin{advise}
            下划线(\_)和美元符号(\$)往往有特殊作用,不建议在自己命名的变量中使用。
        \end{advise}
    }
}

ECMAScript 标识符使用驼峰大小写形式。

\subsubsection{注释}
ECMAScript 采用 C 语言风格的注释。
\begin{itemize}
    \item 单行注释:
    
    \begin{JavaScript}
    // 当行注释
    \end{JavaScript}
    \item 多行注释(块注释):
    \begin{JavaScript}
    /*多行
    注释*/
    \end{JavaScript}
\end{itemize}

\subsubsection{严格模式}

ES5 增加了严格模式,用于处理之前版本(ES3)的一些不规范书写问题。启用严格模式需要在脚本头部加上这一行:

\begin{JavaScript}
"use strict"
\end{JavaScript}

也可以单独在一个函数中执行严格模式:

\begin{JavaScript}
function doSomething() {
    "use strict";
    // 函数体
}
\end{JavaScript}

严格模式会影响到 JavaScript 执行的方方面面,后文会经常出现这个概念。所有现代浏览器都支持严格模式。

\subsubsection{语句}

ECMAScript 中的语句和 Java 几乎一样。他理论上有几种宽松的写法,但并不推荐这样用。

\begin{itemize}
    \item ECMAScript 允许省略分号\footnote{可能影响性能。同时降低可读性。}。
    \item 单条控制语句可以省略括号\footnote{影响不大,我经常这样干。}。
\end{itemize}

\subsection{关键字与保留字}

JavaScript 有很多关键字,关键字不能用作标识符或属性名,具体的关键字视 ES 版本而定,这里不一一列举\footnote{参考文章:\url{https://zhuanlan.zhihu.com/p/257105802}}。

此外,规范中还有一些未来的保留字,同样不能作为标识符或属性名使用。这些保留字多是其他语言中的关键字,有一定其他语言基础的人一般都不会用到。

\subsection{变量}

ECMAScript 变量视松散类型,意思是变量可以用于保存任何类型的数据。共有3个关键字可以声明变量: \texttt{var, let, const}。其中 \texttt{let, const} 只有在 ES6 级之后的版本才可以使用。

\subsubsection{\texttt{var} 关键字}

\texttt{var} 关键字是最原始的声明变量关键字。用法如下:

\begin{JavaScript}
var message = "hi";
\end{JavaScript}

如果没有赋值,仅声明变量,变量会保存一个特殊值 \texttt{undefined}\footnote{下一节讨论。}。这里 \texttt{message} 被定义为一个保存字符串值的变量,但这样初始化变量不会将它标识为字符串类型,只是一个简单的赋值而已。

此外,虽然 ECMAScript 允许变量重新赋值,但是并不推荐这样使用。

\noindent\textbf{\texttt{var} 声明作用域}

使用 \texttt{var} 操作符定义的变量会成为包含它的函数的局部变量。在函数外是不起作用的,这与许多面向对象的高级语言类似。

\begin{JavaScript}
function test() {
    var message = "hi";
}
console.log(message)    // 出错!
\end{JavaScript}

如果我们需要在函数创建一个全局变量,只需要省略 \texttt{var} 关键字。

\begin{JavaScript}
function test() {
    message = "hi";
}
console.log(message)    // 正确
\end{JavaScript}

\fbox{
    \parbox{0.87\textwidth}{
        \begin{warning}
            虽然省略 \texttt{var} 操作符可以定义全局变量,但是不建议这么做。绝大多数面向对象的高级语言也不建议这么做。
        \end{warning}
    }
}

如果要定义多个变量,可以在一条语句中用逗号分隔各个变量\footnote{这种做法也不建议使用。}。