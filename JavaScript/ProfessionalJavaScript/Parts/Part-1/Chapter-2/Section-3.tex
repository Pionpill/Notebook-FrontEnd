\chapter{JavaScript 基础语法}
\section{语言基础}
\subsection{语法}

ECMAScript 语言很大程度上借鉴了 C 语言,作为 C 语言系中的一门,它和 Java 与许多相似之处,比如区分大小写。

\subsubsection{标识符}

JavaScript 对标识符组成规定如下:
\begin{itemize}
    \item 第一个字符必须是一个字母,下划线(\_)或美元符号(\$)。
    \item 剩下的字符可以是字母,数字,下划线(\_)或美元符号(\$)。
\end{itemize}

标识符中的字符是可扩展的 ASCII 中的字母,也可以是 Unicode 的字母字符,但不建议使用 Unicode 字符。

\fbox{
    \parbox{0.87\textwidth}{
        \begin{advise}
            下划线(\_)和美元符号(\$)往往有特殊作用,不建议在自己命名的变量中使用。
        \end{advise}
    }
}

ECMAScript 标识符使用驼峰大小写形式。

\subsubsection{注释}
ECMAScript 采用 C 语言风格的注释。
\begin{itemize}
    \item 单行注释:
    
    \begin{JavaScript}
    // 当行注释
    \end{JavaScript}
    \item 多行注释(块注释):
    \begin{JavaScript}
    /*多行
    注释*/
    \end{JavaScript}
\end{itemize}

\subsubsection{严格模式}

ES5 增加了严格模式,用于处理之前版本(ES3)的一些不规范书写问题。启用严格模式需要在脚本头部加上这一行:

\begin{JavaScript}
"use strict"
\end{JavaScript}

也可以单独在一个函数中执行严格模式:

\begin{JavaScript}
function doSomething() {
    "use strict";
    // 函数体
}
\end{JavaScript}

严格模式会影响到 JavaScript 执行的方方面面,后文会经常出现这个概念。所有现代浏览器都支持严格模式。

\subsubsection{语句}

ECMAScript 中的语句和 Java 几乎一样。他理论上有几种宽松的写法,但并不推荐这样用。

\begin{itemize}
    \item ECMAScript 允许省略分号\footnote{可能影响性能。同时降低可读性。}。
    \item 单条控制语句可以省略括号\footnote{影响不大,我经常这样干。}。
\end{itemize}

\subsection{关键字与保留字}

JavaScript 有很多关键字,关键字不能用作标识符或属性名,具体的关键字视 ES 版本而定,这里不一一列举\footnote{参考文章:\url{https://zhuanlan.zhihu.com/p/257105802}}。

此外,规范中还有一些未来的保留字,同样不能作为标识符或属性名使用。这些保留字多是其他语言中的关键字,有一定其他语言基础的人一般都不会用到。

\subsection{变量}

ECMAScript 变量视松散类型,意思是变量可以用于保存任何类型的数据。共有3个关键字可以声明变量: \texttt{var, let, const}。其中 \texttt{let, const} 只有在 ES6 级之后的版本才可以使用。

\subsubsection{\texttt{var} 关键字}

\texttt{var} 关键字是最原始的声明变量关键字。用法如下:

\begin{JavaScript}
var message = "hi";
\end{JavaScript}

如果没有赋值,仅声明变量,变量会保存一个特殊值 \texttt{undefined}\footnote{下一节讨论。}。这里 \texttt{message} 被定义为一个保存字符串值的变量,但这样初始化变量不会将它标识为字符串类型,只是一个简单的赋值而已。

此外,虽然 ECMAScript 允许变量重新赋值,但是并不推荐这样使用。

\noindent\textbf{\texttt{var} 声明作用域}

使用 \texttt{var} 操作符定义的变量会成为包含它的函数的局部变量。在函数外是不起作用的,这与许多面向对象的高级语言类似。

\begin{JavaScript}
function test() {
    var message = "hi";
}
console.log(message)    // 出错!
\end{JavaScript}

如果我们需要在函数创建一个全局变量,只需要省略 \texttt{var} 关键字。

\begin{JavaScript}
function test() {
    message = "hi";
}
console.log(message)    // 正确
\end{JavaScript}

\fbox{
    \parbox{0.87\textwidth}{
        \begin{warning}
            虽然省略 \texttt{var} 操作符可以定义全局变量,但是不建议这么做。绝大多数面向对象的高级语言也不建议这么做。
        \end{warning}
    }
}

如果要定义多个变量,可以在一条语句中用逗号分隔各个变量\footnote{这种做法也不建议使用。}。

\noindent\textbf{\texttt{var} 声明提升}

下面 JavaScript 代码不会报错。这是因为使用 \texttt{var} 关键字声明的变量会自动提升到函数作用域顶部:

\begin{JavaScript}
function foo() {
    console.log(age);
    var age = 26;
}
\end{JavaScript}

上述代码等价于:

\begin{JavaScript}
function foo() {
    var age;
    console.log(age);
    age = 26;
}   
\end{JavaScript}

这就是所谓的 "提升"(hoist),也就是把所有变量声明都拉到函数作用域的顶部。此外,反复多次使用 \texttt{var} 声明同一个变量也没有问题。

\begin{JavaScript}
function foo() {
    var age = 16;
    var age = 26;
    var age = 36;
    console.log(age);
}
\end{JavaScript}

\subsubsection{\texttt{let} 声明}

\texttt{let} 和 \texttt{var} 作用差不多, 设计 \texttt{let} 就是为了在大部分情况下取代 \texttt{var}。它们最明显的区别是, \texttt{let} 声明的范围是块作用域,而 \texttt{var} 是函数作用域。

\begin{JavaScript}
if (true) {
    var name = 'Matt';
    console.log(name);
}
console.log(name);  // 正确
\end{JavaScript}

\begin{JavaScript}
if (true) {
    let name = 'Matt';
    console.log(name);
}
console.log(name);  // 错误
\end{JavaScript}

在这里, \texttt{age} 变量 之所以不能在 \texttt{if} 块之外被引用,是因为它的作用域仅限于该块内部。块作用域是函数作用域的子集,因此适用于 \texttt{var} 的作用域同样适用于 \texttt{let}。

\texttt{let} 也不允许同一个块作用域中出现冗余声明。这样会报错:

\begin{JavaScript}
var name;
var name;   // 出现两次,无误

let age;
let age;    // 出现两次,SyntaxError
\end{JavaScript}

此外,对声明冗余的报错,不会因混用 \texttt{let} 和 \texttt{var} 而受影响。这两个关键字声明的并不是不同类型的变量,它们只是指出变量在相关作用域如何存在。

\noindent\textbf{暂时性死区}

\texttt{let} 和 \texttt{var} 的另一个重要区别是: 使用\texttt{let}声明的变量不会再作用域中被提升。

在解析代码时, JavaScript 引擎会注意到块后面的 \texttt{let} 声明,只不过在此之前不能以任何方式来引用未声明的变量。在 \texttt{let} 声明之前的执行瞬间被称为 ``暂时性死区'',在此阶段引用任何后面才声明的变量都会抛出 \texttt{ReferenceError}。

\noindent\textbf{全局声明}

与 \texttt{var} 关键字不同,使用 \texttt{let} 在全局作用域中声明的变量不会称为 \texttt{windows} 对象的属性( \texttt{var} 声明会 )。

\begin{JavaScript}
var name = 'Matt';
console.log(window.name);   // 'Matt'

let age = 26;
console.log(window.age);    // undefined
\end{JavaScript}

不过 \texttt{let} 声明仍然是在全局作用域中发生的。相应变量会在页面的生命周期内存续。因此,为了避免 \texttt{SyntaxError},必须确保页面不会重复声明同一个变量。

\noindent\textbf{条件声明}

在使用 \texttt{var} 声明变量时,由于声明被提升, JavaScript 引擎会自动将多余的声明在作用于顶部合并。因为 \texttt{let} 的作用域是块,所以不可能检查前面是否已经使用 \texttt{let} 声明郭同名变量,同时也就不可能在没有声明的情况下声明它。

\begin{JavaScript}
<script>
    var name = 'Pionpill';
    let age = 22;
</script>

<script>
    var name = 'Matt';
    // 没有问题,同时被作为一个提升声明来处理
    // 不需要检查之前是否声明过同名变量
    let age = 23;
    // age 之前声明过,报错
</script>
\end{JavaScript}

使用 \texttt{try/catch} 语句或 \texttt{typeof} 操作符也不能解决,因此条件快中 \texttt{let} 声明的作用域仅限于该块。

\noindent\textbf{\texttt{for} 循环中的 \texttt{let} 声明}

在 \texttt{let} 出现之前, \texttt{for} 循环定义的迭代变量会渗透到循环体外部:

\begin{JavaScript}
for(var i=0; i<5; i++) {
    // 循环逻辑
}
console.log(i);     // 5
\end{JavaScript}

改用 \texttt{let} 后就不会出现这个问题:

\begin{JavaScript}
for(let i=0; i<5; i++) {
    // 循环逻辑
}
console.log(i);     // ReferenceError
\end{JavaScript}

此外,由于 \texttt{var} 对应的是函数作用域,在 \texttt{for} 循环,\texttt{for-in},\texttt{for-of} 循环中会出现种种意想不到的问题。

\subsubsection{\texttt{const} 声明}

\texttt{const} 的行为与 \texttt{let} 基本相同,唯一一个重要的区别是它声明的变量必须同时初始化变量,且尝试修改 \texttt{const} 声明的变量会报错。

此外, \texttt{const} 变量不可变本质上是指其引用不可变,并不是引用的对象不可变\footnote{这和 Python 的元组处理方式相同}。

\begin{JavaScript}
const name = "Pionpill";
name = "Matt";      // 报错

const person = {};
person.name = "Matt";   // 正确
\end{JavaScript}

此外,不能用 \texttt{const} 来声明迭代变量(应为迭代变量会自增):
\begin{JavaScript}
for (const i=0; i<10; i++) {}   // TypeError
\end{JavaScript}

不过,如果只想用 const 声明一个不会被修改的 \texttt{for} 循环变量,那是可以的。也就是说,每次迭代只是创建一个新变量。这对 \texttt{for-in},\texttt{for-of} 循环特别有意义:

\begin{JavaScript}
let i = 0;
for(const j=7; i<5; ++i) {
    console.log(j);
}
// 7,7,7,7,7

for(const key in {a:1,b:2}) {
    console.log(key);
}
// a,b

for (const value of [1,2,3,4,5]) {
    console.log(value);
}
// 1,2,3,4,5
\end{JavaScript}

\subsubsection{如何声明变量}

ES6 加入 \texttt{let} 和 \texttt{const} 关键字,其主要目的就是为了取代 \texttt{var} 关键字。由于 \texttt{var} 声明的变量会出现作用域,升格等问题,它已经在实践中被弃用了。至于为什么还要花大段时间讲 \texttt{var},自然是为了看得懂一些远古代码。

至于 \texttt{const} 关键字,由于浏览器会保持它不变,无论是静态分析还是处理效率上,都要优于 \texttt{let}。当然,使用最多的还是 \texttt{let} 关键字。

总而言之,\textbf{不要使用 \texttt{var},优先使用 \texttt{const}, \texttt{let} 次之。}

\subsection{数据类型}

ECMAScript 有 6 中简单数据类型(也称原始类型): \texttt{Undefined,Null,Boolean,Number,String} 和 \texttt{Symbol}(ES6 新增)。还有一种复杂的数据类 \texttt{Object}。 \texttt{Object} 是一种无序名值对的集合。

此外,ECMAScript 不能定义自己的数据类型,所有值都可以用上述 7 中数据类型之一来表示。ECMAScript 的数据类型十分灵活,一种数据类型可以当作多种数据类型来使用。

\subsubsection{\texttt{typeof} 操作符}

对一个值使用 \texttt{typeof} 操作符会返回下列字符串之一:
\begin{itemize}
    \item \texttt{"undefined"}: 值未定义;
    \item \texttt{"boolean"}: 布尔值;
    \item \texttt{"string"}: 字符串;
    \item \texttt{"number"}: 数值;
    \item \texttt{"object"}: 对象(不是函数)或 \texttt{null} ;
    \item \texttt{"function"}: 函数;
    \item \texttt{"symbol"}: 符号;
\end{itemize}

\texttt{typeof} 在某些情况下返回的结果可能会让人费解。比如,调用 \texttt{typeof(null)} 返回的是 \texttt{"object"}。这是因为特殊值 \texttt{null} 被认为是一个对空对象的引用。

\fbox{
    \parbox{0.87\textwidth}{
        \begin{notice}
            严格来讲,函数在 ECMAScript 中被认为是对象,并不代表一种数据类型。可是,函数也有自己特殊的属性。为此,就有必要通过 \texttt{typeof} 操作符来区分函数和其他对象。
        \end{notice}
    }
}

\subsubsection{\texttt{Undefined} 类型}

\texttt{Undefined} 类型只有一个值,就是特殊值 \texttt{undefined}。当使用 \texttt{var} 或 \texttt{let} 声明了变量但没有初始化时,就相当于给变量赋予了 \texttt{undefined} 值。

\begin{JavaScript}
let message;
console.log(message == undefined);  // true
\end{JavaScript}

我们也可以指定变量值为 \texttt{undefined},与未初始化效果一样。

\begin{JavaScript}
let message = undefined;
console.log(message == undefined);  // true
\end{JavaScript}

\fbox{
    \parbox{0.87\textwidth}{
        \begin{notice}
            虽然可以显式地将 \texttt{undefined} 赋给某个值。但是不要这样做,字面量 \texttt{undefined} 主要是用于比较。而且在 ES3 之前没有  \texttt{undefined}。增加这个特殊值主要是用于和 \texttt{null} 区别。
        \end{notice}
    }
}

与 \texttt{undefined} 字面意义不同的是,它并不表示未声明变量,而表示没有引用对象。

\begin{JavaScript}
let message;

console.log(message);   // "undefined"
console.log(name);      // 报错
\end{JavaScript}

对于未声明的变量,只能执行一个有用的操作,就是对它调用 \texttt{typeof}。需要注意的是对未声明的变量和未初始化的变量调用 \texttt{typeof} 运算都会返回 \texttt{"undefined"}。

\begin{JavaScript}
let message 

console.log(typeof message);   // "undefined"
console.log(typeof name);      // "undefined"
\end{JavaScript}

\fbox{
    \parbox{0.87\textwidth}{
        \begin{advise}
            都返回 \texttt{"undefined"} 往往会让人摸不着头脑,到底是那种情况。因此建议在声明变量的同时一定要进行初始化。这样,当返回 \texttt{"undefined"} 时,就会知道那是因为给定的变量尚未声明。
        \end{advise}
    }
}

\subsubsection{\texttt{Null} 类型}

\texttt{Null} 类型同样只有一个值,即 \texttt{null}。逻辑上讲, \texttt{null} 值表示一个空对象指针,这也是给 \texttt{typeof} 传一个 \texttt{null} 会返回 \texttt{"object"} 的原因。

\begin{JavaScript}
let car = mull;
console.log(typeof car)     // "object"
\end{JavaScript}

在定义将来要保存对象值的变量时,建议使用 \texttt{null} 来初始化,不要使用其他值。这样,只要检查这个变量是不是 \texttt{null} 就可以直到这个变量是否在后来被重新赋予了一个对象的引用。

\texttt{undefined} 是由 \texttt{null} 派生而来的,因此 ECMA 将它们定义为表面上相等。

\begin{JavaScript}
console.log(undefined == null);     // true
\end{JavaScript}

有别于 \texttt{undefined} , 如果要保存未来的对象,可以用 \texttt{null} 来进行填充。

此外, \texttt{undefined} 和 \texttt{null} 都是假值\footnote{在判定语句中等价于 \texttt{false}。 }。

\subsubsection{\texttt{Boolean} 类型}

\texttt{Boolean} 类型有两个字面值: \texttt{true} 和 \texttt{false}。 不同于数值, \texttt{true} 不等于1, \texttt{false} 不等于0。

虽然布尔值只有两个,但是其他所有 ECMAScript 类型的值都有对应的布尔值的等价形式。要将一个其他类型的值转换为布尔值,可以调用特定的 \texttt{Boolean()} 转换函数。

\begin{table}[H]
    \centering
    \small
    \caption{基本数据类型与布尔型之间的转换}
    \label{table:基本数据类型与布尔型之间的转换}
    \setlength{\tabcolsep}{18mm}
    \begin{tabular}{c|cc}
        \toprule
        \textbf{数据类型} & \textbf{转换为 \texttt{true}} & \textbf{转换为 \texttt{false}} \\
        \midrule
        Boolean & true & false\\
        String & 非空串 & 空串 \\
        Number & 非零数值 & 0,NAN \\
        Object & 任意对象 & null \\
        Undefined & N/A & undefined \\
        \bottomrule
    \end{tabular}
\end{table}

\subsubsection{\texttt{Number} 类型}

\texttt{Number} 类型使用 IEEE 754 格式表示整数和浮点数(双精度浮点数)。

\noindent\textbf{整型}

其中整数常用的有有三种写法:十进制,八进制\footnote{严格模式下无效。},十六进制。其中八进制和十六进制表示法在数字前分别加 0 和 0x:

\begin{JavaScript}
let intNum = 55;    // 十进制
let octNum = 070;   // 八进制 56
let hexNum = 0xA;   // 十六进制 10
\end{JavaScript}

如果字面量中包含的数字超出了应有的范围,会忽略前缀的零。

\begin{JavaScript}
let octNum = 079;   // 无效的八进制,当成 79 处理
\end{JavaScript}

\fbox{
    \parbox{0.87\textwidth}{
        \begin{notice}
            JavaScript 中的正零和负零是等同的。
        \end{notice}
    }
}

\noindent\textbf{浮点型}

要定义浮点型,数值中必须包含小数点,且小数点后至少有一个数字。虽然小数点前的数字不是必须的,但还是建议加上。

\begin{JavaScript}
let floatNum1 = 1.1;
let floatNum2 = 0.1;
let floatNum2 = .1;     // 有效,但不建议这样用
\end{JavaScript}

出于优化考虑, ECMAScript 会想方设法把值转换为整数。如果值本省是整数,那么它会被自动转换为整型值。

\begin{JavaScript}
let floatNum1 = 1.;     // 小数点后没有数字,当作整型1
let floatNum2 = 10.0;   // 小数点后是零,当作整型10
\end{JavaScript}

此外,ECMAScript 也可以使用科学计数法,在一定情况下会将小数转换为科学计数法,其精度最高可达小数点后 17 位。

\begin{JavaScript}
let floatNum1 = 3e-7;   // 0.0000007
\end{JavaScript}

此外,和大多数高级语言一样,为了避免精度误差,请不要对小数运用比较运算。

\noindent\textbf{值得范围}

ECMAScript 可表示的最大最小值分别存储在了以下两个变量中(大多数浏览器中值与下列相同):

\begin{itemize}
    \item \texttt{Number.MIN\_VALUE}: 5e-324
    \item \texttt{Number.MAX\_VALUE}: 1.797e+308
\end{itemize}

如果计算的值超出了这个范围,会自动被转换为特殊的 \texttt{Infinity}(无穷值),负值则为 \texttt{-Infinity}。如果计算返回无穷,则该值将不能再进一步用于任何计算。可以用 \texttt{isFinite()} 函数判断是否为无穷值。

\begin{JavaScript}
let result = Number.MAX_VALUE + Number.MAX_VALUE;
console.log(isFinite(result));      // false
\end{JavaScript}

此外,可以使用 \texttt{Number.NEGATIVE\_INFINITY} 和 \texttt{Number.POSITIVE\_INFINITY} 获取正负 \texttt{Infinity}。

\noindent\textbf{\texttt{NaN}}

\texttt{NaN} 用于表示不是数值(Not a Number),一般用于表示本要返回数值得操作失败了。

造成 \texttt{NaN} 得原因有很多,比如分子为零得运算。此外任何涉及 \texttt{NaN} 得操作始终返回 \texttt{NaN}。其次, \texttt{NaN} 不等于包括 \texttt{NaN} 在内得任何值。

\begin{JavaScript}
console.log(NaN == NaN);    // false
\end{JavaScript}

为了判断 \texttt{NaN},JavaScript 提供了 \texttt{isNaN()} 函数。这个函数非常灵活,任何可以转化为数值类型得值都会返回 \texttt{true}:

\begin{JavaScript}
console.log(isNaN(NaN));        // true
console.log(isNaN(10));         // false,10是数值
console.log(isNaN("10"));       // false,"10"可以转换为数值
console.log(isNaN("blue"));     // true,不可以转换为数值
console.log(isNaN(true));       // false,可以转换为数值1
\end{JavaScript}

\noindent\textbf{数值转换}

有3个函数可以将非数值转换为数值: \texttt{Number(),parseInt(),parseFloat()}。 其中 \texttt{Number()} 是转型函数,可用于任何数据类型。后两个函数主要用于将字符串转换为数值。

\texttt{Number()} 函数基于如下规则执行转换:

\begin{itemize}
    \item 布尔值: \texttt{true} 转为1, \texttt{false} 转为 0。
    \item 数值: 直接返回。
    \item \texttt{null}: 返回0。
    \item \texttt{undefined}: 返回 \texttt{NaN}。
    \item 字符串,有如下规则:
    \begin{itemize}
        \item 整数字符串,包含字符前有加减号得情况,转换为一个对应的十进制数值。忽略最前面的零。
        \item 浮点格式:转换为浮点数。
        \item 十六进制:例如 \texttt{"0xf"},转换为对应的十进制数值。
        \item 空字符串:转为 0。
        \item 其他情况:\texttt{NaN}。
    \end{itemize}
    \item 对象: 调用 \texttt{valueOf()} 方法,并按照上述规则转换返回的值。如果转换结果为 \texttt{NaN},则调用 \texttt{toString()} 方法,再按照转换字符串的规则转换。
\end{itemize}

考虑到 \texttt{Number()} 函数转换字符串时相对复杂且有点反常规,通常需要在得到整数时可以优先使用 \texttt{parseInt()} 函数。

\texttt{parseInt()} 函数更专注于字符串是否包含数值模式。字符串最前面的空格会被忽略,从第一个非空格字符串开始转换。如果第一个字符串不是数值字符,加号,减号。 \texttt{parseInt()} 立即返回 \texttt{NaN}。这意味着空字符串也返回 \texttt{NaN}。如果第一个字符是数值字符,加号,减号;则继续依次检测每个字符,直到字符串末尾,或碰到非数值字符。

\begin{JavaScript}
parseInt("1234blue");   // 1234
parseInt("22.5");       // 22
\end{JavaScript}

假设字符串第一个字符是数值字符,\texttt {parseInt()} 函数也能识别不同的整数格式(十六进制,八进制,十进制)。如果字符以 ``0'' 开头,且紧跟着数值字符,在非严格模式下会被某些实现解释为八进制整数。

\begin{JavaScript}
parseInt("0xA");        // 10
parseInt("070");        // 56(视浏览器而定)
\end{JavaScript}

不同的数值格式很容易混淆,因此 \texttt{parseInt()} 函数也接收第二个参数,用于指定进制数,例如:

\begin{JavaScript}
parseInt("0xAF",16);    // 175
parseInt("AF",16);      // 175
parseInt("AF");         // NaN
\end{JavaScript}

通过第二个参数,可以极大扩展转换后获得的结果类型:

\begin{JavaScript}
parseInt("10",2);       // 二进制
parseInt("10",8);       // 八进制
parseInt("10",10);      // 十进制
parseInt("10",16);      // 十六进制
\end{JavaScript}

因为不传底数参数相当于让 \texttt{parseInt()} 函数自己判断如何解释,所以应始终传第二个参数,这个参数多数情况下是 10。

\fbox{
    \parbox{0.87\textwidth}{
        \begin{advise}
            \texttt{parseInt()} 函数与 \texttt{Number()} 函数的主要区别在于:1.对空字符串的处理。 2.字符串后面非数值字符的处理。3.八进制处理(视浏览器决定)。
        \end{advise}
    }
}

\subsubsection{\texttt{String} 类型}

\texttt{String} 数据类型表示零个或多个 16 为 Unicode 字符序号。字符串可以用双引号(\texttt{"}),单引号(\texttt{'}),反引号(\texttt{`}) 标志。但引号必须同队出现,不同类型引号不可以随意匹配。

注意,ECMAScript 中的字符串是不可变的。

\noindent\textbf{转换为字符串}

JavaScript 提供了两种方式转换字符串。首先是几乎所有值都有的 \texttt{toString()} 方法。这个方法唯一的用途就是返回当前值的字符串等价物。

\texttt{toString()} 方法可用于数值,布尔值,对象和字符串值(字符串值返回自身的一个副本)。 \texttt{null} 和 \texttt{undefined} 值没有 \texttt{toString()} 方法。

\begin{JavaScript}
let age = 11;
let ageString = age.toString();         // 字符串 "11"
let found = true;
let foundString = found.toString();     // 字符串 "true"
\end{JavaScript}

多数情况下, \texttt{toString()} 不接收任何参数。不过,在对数值调用这个方法时,\texttt{toString()} 可以接收一个底数参数,即以什么底数来输出数值的字符串表示。默认为十进制(即参数为10)。

\begin{JavaScript}
let num = 10;
console.log(num.toString());    // "10"
console.log(num.toString(2));   // "1010"
console.log(num.toString(8));   // "12"
console.log(num.toString(16));  // "a"
\end{JavaScript}

如果不确定一个值是不是 \texttt{null} 或 \texttt{undefined} ,可以使用 \texttt{String()} 转型函数,它始终会返回表示相应类型值的字符串,其处理规则如下:

\begin{itemize}
    \item 如果只有 \texttt{toString()} 方法,则调用该方法并返回结果。
    \item 如果值是 \texttt{null},则返回 \texttt{null}。
    \item 如果值是 \texttt{undefined},则返回 \texttt{undefined}。
\end{itemize}

\noindent\textbf{模板字面量}

ES6 新增了使用模板字面量定义字符串的能力。与使用单引号或双引号不同,模板字面量保留换行字符,可以跨行定义字符串。

\begin{JavaScript}
let multiLine1 = "first lin\nsecond line";
let multiLine1 = "first line
second line";

console.log(multiline1);
// first line
// second line

console.log(multiline2);
// first line
// second line
\end{JavaScript}

\noindent\textbf{字符串插值}

模板字面量最常用的一个特性是支持字符串插值,也就是可以在一个连续定义中插入一个或多个值。技术上讲,模板字面量不是字符串,而是一种特殊的 JavaScript 语法表达式,只不过求值后得到的是字符串。模板字面量在定义时立即求值并转换为字符串实例,任何插入的变量也会从它们最接近的作用域中取值。

字符串插值通常在 \texttt{\$\{\}} 中使用一个 JavaScript 表达式实现:

\begin{JavaScript}
let value = 5;
let exponent = 'second';
// 之前的写法
let interpolatedString = value + ' to the ' + exponent + ' power is ' + (value * value);
// 现在的写法
let interpolatedString = '${value} to the ${exponent} power is ${value*value}'; // $ 写法
\end{JavaScript}

所有插入的值都会使用 \texttt{toString()} 强制转型为字符串,而且任何 JavaScript 表达式都可以用于插值。嵌套的模板字符串无需转义:

\begin{JavaScript}
function capitalize(word) {
    return '${word[0].toUpperCase()} ${word.slice[1]}'
}
console.log('${capitalize('hello')}, ${capitalize('world')}!')  // Hello, World!
\end{JavaScript}

\noindent\textbf{模板字面量标签函数}

模板字面量也支持定义标签函数,通过标签函数可以自定义插值行为。标签函数会接收被插值记号分隔后的模板和对每个表达式求值的结果。

标签函数本身是一个常规函数,通过前缀到模板字面量来应用自定义行为。标签函数接收到的参数依次是原始字符串数组和对每个表达式求值的结果。这个函数的返回值是对模板字面量求值得到的字符串。

\begin{JavaScript}
let a = 6;
let b = 9;

function simpleTag(strings, aValExpression, bValExpression, sumExpression) {
    console.log(strings);
    console.log(aValExpression);
    console.log(bValExpression);
    console.log(sumExpression);
    return 'foobar';
}

let untaggedResult = '${a} + ${b} == ${a+b}';
let taggedResult = simpleTag'${a} + ${b} == ${a+b}';
// ["","+","=",""]
// 6
// 9
// 15

console.log(untaggedResult);    // "6 + 9 = 15"
console.log(taggedResult);      // "foobar"
\end{JavaScript}

因为表达式参数的数量是可变的,所以通常应该使用剩余操作符将它们收集到一组数组中:

\begin{JavaScript}
let a = 6;
let b = 9;
    
function simpleTag(strings, ...expressions) {
    console.log(strings);
    for (const expression of expressions) {
        console.log(expression);
    }
    return 'foobar';
}
    
let taggedResult = simpleTag'${a} + ${b} == ${a+b}';
// ["","+","=",""]
// 6
// 9
// 15
    
console.log(taggedResult);      // "foobar"
\end{JavaScript}

对于有 n 个插值的模板字面量,传给标签函数的表达式参数的个数始终是 n。而传给标签函数的第一个参数所包含的字符串个数则始终是 n+1。因此,如果你想把这些字符串和对表达式求值的结果拼接起来作为默认返回的字符串,可以这样做:

\begin{JavaScript}
    let a = 6;
    let b = 9;
        
    function zipTag(strings, ...expressions) {
        return strings[0] + expressions.map((e,i) => '${e}${strings[i+1]}').join('');
    }
        
let taggedResult = zipTag'${a} + ${b} == ${a+b}';
        
console.log(taggedResult);      // "6 + 9 = 15"
\end{JavaScript}

\noindent\textbf{原始字符串}

使用模板字面量也可以直接获取原始的模板字面量内容(如换行符或 Unicode字符),而不是被转换后的字符表示。为此,可以使用默认的 \texttt{String.raw} 标签函数:

\begin{JavaScript}
console.log('\u00A9');              // ©
console.log(String.raw'\u00A9');    // \u00A9

console.log(String.raw'first line
second line')     
// first line
// second line 对实际的换行符不起作用
\end{JavaScript}

\subsubsection{\texttt{Symbol} 符号类型}

\texttt{Symbol} 是 ES6 新增的数据类型。符号是原始值,是唯一的,不可变的。符号的用途是确保对象属性使用唯一标识符,不会发生属性冲突的危险。

\noindent\textbf{符号的基本用法}

符号要使用 \texttt{Symbol()} 函数初始化,也可以传入一个字符串参数作为符号的描述,将来可以通过这个字符串来调试代码。但是,这个字符串参数与符号定义或标识完全无关:

\begin{JavaScript}
let firstSymbol = Symbol();
let secondSymbol = Symbol();

let thirdSymbol = Symbol("foo");
let forthSymbol = Symbol("foo");

console.log(firstSymbol == secondSymbol);   // false
console.log(thirdSymbol == forthSymbol);    // false

console.log(firstSymbol);    // Symbol()
console.log(thirdSymbol);    // Symbol(foo)
\end{JavaScript}

\texttt{Symbol} 函数不能与 \texttt{new} 关键字一起作为构造函数使用。这样做是为了避免创建符号包装对象,如果一定要做可以使用 \texttt{Object()} 函数:

\begin{JavaScript}
let mySymbol = Symbol;
let myWrappedSymbol = Object(mySymbol);
console.log(typeof myWrappedSymbol)
\end{JavaScript}

\noindent\textbf{使用全局符号注册表}

如果运行时的不同部分需要共享和重用符号实例,那么可以用一个字符串作为键,在全局符号注册表中创建并重用符号。为此,需要使用 \texttt{Symbol.for()} 方法:

\begin{JavaScript}
let fooGlobalSymbol = Symbol.for('foo');
console.log(typeof fooGlobalSymbol);    // symbol
\end{JavaScript}

\texttt{Symbol.for()} 对每个字符串健都执行等幂操作。第一次使用某个字符串调用时,它会检查全局运行时注册表,发现不存在对应的符号,于是就会生成一个新符号实例并添加到注册表中。后续使用相同字符串的调用同样会检查注册表,大仙存在与该字符串对应的符号,然后就会返回该符号实例。

\begin{JavaScript}
let fooGlobalSymbol = Symbol.for('foo');        // 创建新符号
let otherFooGlobalSymbol = Symbol.for('foo');   // 重用已有符号

console.log(fooGlobalSymbol === otherFooGlobalSymbol);  // true
\end{JavaScript}

在全局注册表中定义的符号和使用 \texttt{Symbol()} 定义的符号并不等同:

\begin{JavaScript}
let localSymbol = Symbol('foo');
let globalSymbol = Symbol.for('foo');

console.log(localSymbol === globalSymbol);  // false
\end{JavaScript}

全局注册表中的符号必须使用字符串键来创建,因此作为参数传给 \texttt{Symbol.for()} 的任何值都会被转换为字符串。此外,注册表中使用的键同时也会被用作符号描述。

\begin{JavaScript}
let emptyGlobalSymbol = Symbol.for();
console.log(emptyGlobalSymbol);     // Symbol(undefined)
\end{JavaScript}

还可以使用 \texttt{Symbol.keyFor()} 来查询全局注册表,这个方法接收符号,返回该全局符号对应的字符串键。如果查询的不是全局符号,则返回 \texttt{undefined} 。

\begin{JavaScript}
let s = Symbol.for('foo');
console.log(Symbol.keyFor(s));      // foo

let s2 = Symbol('bar');
console.log(Symbol.keyFor(s2));     // undefined

Symbol.keyFor(123);                 // TypeError: 123 is not a symbol
\end{JavaScript}

\noindent\textbf{使用符号作为属性}

凡是可以使用字符串或数值作为属性的地方,都可以使用符号。这就包括了对象字面量属性 \texttt{Object.defineProperty()/Object.defineProperties()} 定义的属性。对象字面量只能在计算性语法中使用符号作为属性。

\fbox{
    \parbox{0.87\textwidth}{
        \begin{notice}
            原文 P47-55 页还有更多关于 Symbol 的介绍,但本人觉得,Symbol 作为新的数据类型并不值得这么大篇幅的详细介绍,尤其是在基础篇中,故省略,读者可自行查看原文,但不查看也不会影响后续阅读。
        \end{notice}
    }
}

\subsubsection{\texttt{Object} 类型}

ES 中的对象其实就是一组数据和功能的集合。对象通过 \texttt{new} 操作符后跟对象类型的名称来创建。开发者可以通过创建 \texttt{Object} 类型的实例来创建自己的对象,然后再给对象添加属性和方法。

\begin{JavaScript}
let o = new Object();
let p = new Object;     // 可行,但不推荐
\end{JavaScript}

类似 Java 中的 \texttt{java.lang.Object} ES 中的 \texttt{Object} 也是派生其他对象的基类。 \texttt{Object} 类型的所有属性和方法在派生的对象上同样存在,每个 \texttt{Object} 属性都有如下属性和方法:

\begin{itemize}
    \item \texttt{constructor}: 用于创建当前对象的函数。在前面例子中值是 \texttt{Object()} 函数。
    \item \texttt{hasOwnProperty(propertyName)}: 用于判断当前对象实例上是否存在给定的属性。要检查的属性名必须是字符串或符号。
    \item \texttt{isPrototypeOf(object)}: 用于判断当前对象是否为另一个对象的原型。
    \item \texttt{propertyIsEnumerable(propertyName)}: 用于判断给定的属性是否可以使用 \texttt{for-in} 语句枚举,属性名必须是字符串。
    \item \texttt{toLocalString()}: 返回对象的字符串标识,该字符串反映对象所在的本地化执行环境。
    \item \texttt{toString()}: 返回对象的字符串表示。
    \item \texttt{valueOf()}: 返回对象对应的字符串,数值或布尔值表示,通常与 \texttt{toString()} 的返回值相同。
\end{itemize}

\fbox{
    \parbox{0.87\textwidth}{
        \begin{notice}
            严格来讲,ECMA 中的对象行为不一定适合 JavaScript 中的其他对象,因为还有 BOM,DOM 对象。所以并不是所有对象都继承自 Object 对象。
        \end{notice}
    }
}

\subsection{操作符}

JavaScript 中的基本操作符和 C 语言系中的操作符运算是相同的,相同内容省略。此外,和 Python 一样,JavaScript 是一门十分灵活的语言,比如 \texttt{false == 0} 会返回 \texttt{true},这种数据类型转换在操作符运算中时常会出现。限于操作符类型太多,涉及数据类型转换的操作更多,所以不一一说明,如果遇到问题,读者可通过 node.js 自行检查。

\subsubsection{值得说明的操作符}

\textit{为什么叫值得说明的操作符呢,因为这章讲的太基础了捏,很多都是面向完全无基础的人写的,故直接省略。只写本人觉得有必要说明的。}

\noindent\textbf{递增递减操作符}

和 C/Java 语言中一样的内容省略。这个操作符往往会降低可读性,不建议使用,Python 就完全抛弃了。

ECMA 中的递增递减操作符除了对数值类型起作用,其他类型也起作用:
\begin{itemize}
    \item 字符串,如果是有效的数值形式,则转换为数值进行计算,变量类型由此变成数值类型。
    \item 字符串,如果不是有效的数值形式,则转换为 \texttt{NaN},变量类型由此变成数值类型。
    \item 布尔型,变成对应的 0/1 再进行计算,变量类型变成数值。
    \item 浮点数,加1或减1。
    \item 对象,调用 \texttt{valueOf()} 方法取得可以操作的值。对得到的值采用上述规则。如果是 \texttt{NaN},调用 \texttt{toString()} 并再次应用其他规则。变量类型变成数值类型。
\end{itemize}

\noindent\textbf{指数操作符}

ES7 新增了指数操作符,其语法和 Python 中的指数操作符相同。

\begin{JavaScript}
console.log(Math.pow(3,2));     // 9
cnosole.log(3**2);              // 9
\end{JavaScript}

不止如此,ECMA还提供了指数赋值操作符 \texttt{**+}:

\begin{JavaScript}
let a = 3;
a **= 2;
console.log(a);     // 9
\end{JavaScript}

\noindent\textbf{全等于运算}

ECMA 中的相等(\texttt{==})符号默认会转换操作数,而全等(\texttt{===})符号则不会。

\begin{JavaScript}
console.log("55" == 55);    // true
console.log("55" === 55);   // false
\end{JavaScript}

与全等对应的还有不全等 (\texttt{!==}) 运算。

\fbox{
    \parbox{0.87\textwidth}{
        \begin{advise}
            多数情况下,由于等于运算涉及类型转换问题,更推荐使用全等运算,这有助于保持数据类型的完整性。
        \end{advise}
    }
}

\subsection{语句}

省略 \texttt{if,while,for} 语句的介绍,这三种语句实在是太基础了。

\subsubsection{\texttt{\texttt{for-in} 语句}}

\texttt{for-in} 语句是一种严格的迭代语句,用于枚举对象中的非符号键属性,语法如下:

\begin{JavaScript}
for (property in expression) statement
\end{JavaScript}

下面是一个例子:

\begin{JavaScript}
for (const propName in window) {
    document.write(propName);
}
\end{JavaScript}

这个例子使用 \texttt{for-in} 循环显示了 BOM 对象 \texttt{window} 的所有属性。这里控制语句中的 \texttt{const} 不是必须的,不过为了确保局部变量不被修改,推荐使用 \texttt{const}。

值得说明的是,ECMAScript 中对象的属性是无序的。

\subsubsection{\texttt{for-of} 语句}

\texttt{for-of} 语句是一种严格的迭代语句,用遍历可迭代对象的元素,语法如下:

\begin{JavaScript}
for (property in expression) statement
\end{JavaScript}

下面是示例:

\begin{JavaScript}
for (const el of [2,4,6,8]) {
    document.write(el);
}
\end{JavaScript}

\subsubsection{标签语句}

标签语句用于给语句加标签,语法如下:
\begin{JavaScript}
label: statement
\end{JavaScript}
下面是一个例子:
\begin{JavaScript}
start: for(lei i=0; i<count; i++) {
    console.log(i);
}
\end{JavaScript}

这个语句中的 \texttt{start} 是一个标签,可以在后面通过 \texttt{break} 或 \texttt{continue} 语句引用。标签语句的典型应用场景是嵌套循环。

\subsubsection{\texttt{with} 语句}

\texttt{with} 语句的用途是将代码作用域设置为特定的对象,其语法是:

\begin{JavaScript}
with (expression) statement;
\end{JavaScript}

使用 \texttt{with} 语句的主要场景是针对一个对象反复操作,这时候将代码作用域设置为该对象能提供便利。

\begin{JavaScript}
// 不使用 with 语句
let qs = location.search.substring(1);
let hostname = location.hostname;
let url = location.href;

// 使用 with 语句
with(location) {
    let qs = search.substring(1);
    let hostName = hostname;
    let url = href;
}
\end{JavaScript}

在 \texttt{with} 语句中,每个变量会首先被认为是一个局部变量。如果没有找到局部变量,则会搜索 \texttt{location} 对象,看它是否有同名属性。

\fbox{
    \parbox{0.87\textwidth}{
        \begin{advise}
            严格模式下不允许使用 \texttt{with} 语句,由于 \texttt{with} 语句影响性能且难于调试,所以不建议使用。
        \end{advise}
    }
}

\subsection{函数}

后续章节会更详细地讲解函数,下面是函数的基础语法:

\begin{JavaScript}
function functionName(arg0, arg1, ...) {
    statements
}
\end{JavaScript}

\newpage